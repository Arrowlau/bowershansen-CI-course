%  For slides only
\documentclass{beamer} 
\newcommand{\tcw}{\textcolor{black}}
\newcommand{\mynoteonly}{}
\newcommand{\nottheirhandout}{handout:0}

% % For handout
%\documentclass[handout]{beamer}
\usepackage{pgfpages}
\pgfpagesuselayout{4 on 1}[letterpaper,border shrink=5mm]%\nofiles

\mode<handout>{\setbeamercolor{background canvas}{bg=black!5}}
\newcommand{\tcw}{\textcolor{structure.bg}}
\newcommand{\mynoteonly}{| handout:0}
\newcommand{\nottheirhandout}{handout:0}

%For handout + mynotes
%\documentclass[handout]{beamer}
\usepackage{pgfpages}
%%\pgfpagesuselayout{2 on 1}[letterpaper,border shrink=5mm] 
\setbeameroption{show notes on second screen=left}
\newcommand{\tcw}{\textcolor{black}}
\newcommand{\mynoteonly}{}
\newcommand{\nottheirhandout}{}

\newcommand{\igrphx}[2][width=\linewidth]{\includegraphics[#1]{images/#2}}

\renewcommand{\strut}{\rule{0pt}{3ex}}

% Frame note commands, with time budget updating
\newcounter{timeTotal}
\newcommand{\printUpdateTimeTotal}[1]{ \addtocounter{timeTotal}{#1}
  \textrm{(#1 min budgeted)} \hfill \textbf{Finish by
    \arabic{timeTotal}\ min from start}\\[1ex]}
\newcommand{\tnote}[3][10]{#2 \note{#2: #3 \\[.5ex]} \printUpdateTimeTotal{#1}}
\newcommand{\itnote}[2][10]{\note{ \begin{itemize} #2 \end{itemize}
\vfill \noindent  \printUpdateTimeTotal{#1} }}
\newcommand{\ennote}[2][10]{\note{ \begin{enumerate} #2 \end{enumerate}
\vfill \noindent  \printUpdateTimeTotal{#1} }}
\newcommand{\Note}[2][10]{\note{ #2 \mbox{ }\\ \vfill \noindent  \printUpdateTimeTotal{#1} }}

\newenvironment{Column}[1][.5\linewidth]{\begin{column}{#1}}{\end{column}}

\mode<handout>{\beamertemplatesolidbackgroundcolor{black!5}} 
\mode<article>{\usepackage{fullpage}}
\mode<presentation>
{
  \usetheme{Boadilla}
  % or ...

  \setbeamercovered{transparent}
  % or whatever (possibly just delete it)
}
\usepackage{url}
\usepackage{ulem}

\usepackage[english]{babel}
% or whatever

\usepackage[latin1]{inputenc}
% or whatever

\usepackage{times}
\usepackage[T1]{fontenc}
% Or whatever. Note that the encoding and the font should match. If T1
% does not look nice, try deleting the line with the fontenc.

\AtBeginSubsection[]
{
  \begin{frame}<beamer>
    \frametitle{Outline}
    \tableofcontents[subsectionstyle=show/shaded/hide]
  \end{frame}
}

\AtBeginSection[]
{
  \begin{frame}<beamer>
    \frametitle{Outline}
    \tableofcontents[hideothersubsections,sectionstyle=show/shaded]
  \end{frame}
}


% If you wish to uncover everything in a step-wise fashion, uncomment
% the following command: 

%\beamerdefaultoverlayspecification{<+->}

%USEFUL CODE TEMPLATES:
%\begin{itemize}[<+-| alert@+>]


% \begin{frame}[fragile]


% \begin{columns}
% \column{.4\textwidth}  
%   \begin{itemize}
%   \item<1-| alert@1> Why sample twice?
%   \item<2-| alert@2> Illustrative example.
%   \item<3-| alert@3> PSU's and SSU's.
%   \item<4-| alert@4> Stratification in combination with two-stage cluster sampling.
%   \item<5-| alert@5> Detailed example.
%   \end{itemize}
% \column{.6\textwidth}
% \only<2| handout:0>{
% \includegraphics[width=\textwidth]{nursingHomes}}
% \only<5| handout:1>{
% \includegraphics[width=\textwidth]{hosp_cover}}
% \end{columns}

% FOR INCLUDING R CODE. 
%\usepackage{listings}
%\lstset{language=R}
% \begin{frame}[fragile]
% \frametitle{Some code}
% \begin{lstlisting}
% > plot(myobj)
% > rm(myobj)
% \end{lstlisting}  
% \end{frame}

%\addtocounter{framenumber}{-1}

\author{ICPSR Causal Inference (BH)}
\date{2014}
\usepackage{amsmath,amsthm}
\usepackage{txfonts,wasysym,pifont}
\usepackage{Sweave}
\usepackage{ulem}
\usepackage{textcomp}
\usepackage{versions}
%% amsthm-type theorem environment specifications -- 
%% see amsthdoc.pdf in amscls documentation
\theoremstyle{plain}
\newtheorem{prop}{Proposition}[section]
\newtheorem{lem}[prop]{Lemma}

\newtheorem*{thm}{Proposition}
\newtheorem*{cor}{Corollary}

\theoremstyle{definition}
\newtheorem{defn}{Definition}[section]

\newcommand{\Pdistsym}{P}
\newcommand{\Pdistsymn}{P_n}
\newcommand{\Qdistsym}{Q}
\newcommand{\Qdistsymn}{Q_n}
\newcommand{\Qdistsymni}{Q_{n_i}}
\newcommand{\Qdistsymt}{Q[t]}
\newcommand{\dQdP}{\ensuremath{\frac{dQ}{dP}}}
\newcommand{\dQdPn}{\ensuremath{\frac{dQ_{n}}{dP_{n}}}}
\newcommand{\EE}{\ensuremath{\mathbf{E}}}
\newcommand{\EEp}{\ensuremath{\mathbf{E}_{P}}}
\newcommand{\EEpn}{\ensuremath{\mathbf{E}_{P_{n}}}}
\newcommand{\EEq}{\ensuremath{\mathbf{E}_{Q}}}
\newcommand{\EEqn}{\ensuremath{\mathbf{E}_{Q_{n}}}}
\newcommand{\EEqni}{\ensuremath{\mathbf{E}_{Q_{n[i]}}}}
\newcommand{\EEqt}{\ensuremath{\mathbf{E}_{Q[t]}}}
\newcommand{\PP}{\ensuremath{\mathbf{Pr}}}
\newcommand{\PPp}{\ensuremath{\mathbf{Pr}_{P}}}
\newcommand{\PPpn}{\ensuremath{\mathbf{Pr}_{P_{n}}}}
\newcommand{\PPq}{\ensuremath{\mathbf{Pr}_{Q}}}
\newcommand{\PPqn}{\ensuremath{\mathbf{Pr}_{Q_{n}}}}
\newcommand{\PPqt}{\ensuremath{\mathbf{Pr}_{Q[t]}}}
\newcommand{\var}{\ensuremath{\mathbf{V}}}
\newcommand{\varp}{\ensuremath{\mathbf{V}_{P}}}
\newcommand{\varpn}{\ensuremath{\mathbf{V}_{P_{n}}}}
\newcommand{\varq}{\ensuremath{\mathbf{V}_{Q}}}
\newcommand{\cov}{\ensuremath{\mathbf{Cov}}}
\newcommand{\covp}{\ensuremath{\mathbf{Cov}_{P}}}
\newcommand{\covpn}{\ensuremath{\mathbf{Cov}_{P_{n}}}}
\newcommand{\covq}{\ensuremath{\mathbf{Cov}_{Q}}}

\newcommand{\hatvar}{\ensuremath{\widehat{\mathrm{Var}}}}
\newcommand{\hatcov}{\ensuremath{\widehat{\mathrm{Cov}}}}

\newcommand{\sehat}{\ensuremath{\widehat{\mathrm{se}}}}

\newcommand{\combdiff}[1]{\ensuremath{\Delta_{{z}}[#1]}}
\newcommand{\Combdiff}[1]{\ensuremath{\Delta_{{Z}}[#1]}}

\newcommand{\psvec}{\ensuremath{\varphi}}
\newcommand{\psvecgc}{\ensuremath{\tilde{\varphi}}}


\newcommand{\atob}[2]{\ensuremath{#1\!\! :\!\! #2}}
\newcommand{\stratA}{\ensuremath{\mathbf{S}}}
\newcommand{\stratAnumstrat}{\ensuremath{S}}
\newcommand{\sAsi}{\ensuremath{s}}

\newcommand{\permsd}{\ensuremath{\sigma_{\Pdistsym}}}
\newcommand{\dz}[1]{\ensuremath{d_{z}[{#1}]}}
\newcommand{\dZ}[1]{\ensuremath{d_{Z}[{#1}]}}
\newcommand{\tz}[1]{\ensuremath{t_{{z}}[#1]}}
\newcommand{\tZ}[1]{\ensuremath{t_{{Z}}[#1]}}


\newlength{\tabcolsepadj}
\setlength{\tabcolsepadj}{1.3mm}

%%% NEWBLOCK UNDEFINED BUG
\def\newblock{\hskip .11em plus .33em minus .07em}



\title{Unit 1: Documenting causation with experiments}
% \autho moved to beamer-preamble-*-all.tex
\date{July 2015}

\begin{document}


  \begin{frame}
    \frametitle{Outline \& Readings}

\tableofcontents[subsectionstyle=show/hide/hide]

  \alert{Announcements:}\\
  \begin{itemize}
\item Please fill out an index card w/ your 1st \& last name,
  preferred email,
  laptop?.  Also, discipline, institution, interests. Leave a little space.  Turn in
  at end \textit{if} you're likely to keep attending.
\item Readings for Tues:
   \begin{enumerate}
     \item Kinder \& Palfrey (1993), particularly \S 1.2, 3
    (pp.5--10). (Course \url{umich.instructure.com} site, or \url{http://tinyurl.com/pnqk48n})
  \item Rosenbaum, \textit{Design of Observational Studies} (2009), \S
    2.1. (Course site on \url{umich.instructure.com}.)
   \end{enumerate}
  \item Related courses: LaTeX, today at 5:30; R, starting tomorrow at 5:30.
%  \item Fisher (1935), \S 1--10. later
  \end{itemize}
  \end{frame}

\itnote{
\item Re emails: (1) UM address, if you have it and you use it for UM
  resources; (2) address you gave the s.p., provided that you use it;
  (3) another address if you prefer to receive email there (clearly
  mark!). 
}

\section{Overview}

\begin{frame}{``Causal Inference''}

 \textit{Causal inference} is about design and analysis of randomized and nonrandomized studies, aiming to create meaningful comparisons that justify inferences about causation with varying, but clearly specified, degrees of certainty.  
\pause

\end{frame}

\begin{frame}[label=whatWeWillCoverFr]{Methods we'll study in the course}
  \begin{itemize}
  \item Randomization-based analysis for experiments
  \item Instrumental variables
  \item Natural experiments, quasi-experiments,
    difference-in-differences
  \item Propensity scores, propensity score matching
  \item Regression discontinuity
  \item Interference
  \item Omitted variable sensitivity analysis
  \item \ldots
  \end{itemize}
\end{frame}

\Note{Don't linger, I come back to this slide after Causal Inference
  in Experiments section}

\begin{frame}{Three traditions in quantitative analysis}
\framesubtitle{A brief, tendentious review}

Modern causal inference in the 19th c., emerges over the 20th c, hits
it stride during this century. Dominant strains of quantitative
analysis in social science, by century:

\begin{itemize}
\item[19th] social physics
\item[20th] sampling of populations
\item[21st] causal inference (BH prediction)
\end{itemize}

  
\end{frame}

\section{Causal inference in simple experiments}

\begin{frame}{Centrality of experiment to modern CI}
  \begin{itemize}
  \item Experiments are conceptually and practically central
  \item Not just any experiments, particular types, featuring
    \textit{control groups} and \textit{random assignment}
  \item Central characteristics of an ideal situation
  \item The ideal reveals its merits in a variety of disciplines over
    the 20th c.  
  \end{itemize}
\end{frame}

\subsection{Two influential examples}
\begin{frame}{Example 1a: The Salk vaccine trial}


\igrphx{SalkVtable-rctonly}

\end{frame}
\itnote{
\item Vaccines contentious
\item Geographic variation
\item year to year variation
\item randomized
\item double blinded
\item large, national sample
\item Exercise: Abstract a relevant 2x2 table
}

\begin{frame}{The other Salk trial}

\igrphx{SalkVtable-full}
\end{frame}
\itnote{
\item Explain differences
\item both versions are experiments in the everyday sense
\item both control for age and physical locale
\item large size, presence of these controls lends legitimacy to both
\item Demonstration of advantage of random assigment per se
\item We'll see how to get the ``right answer'', despite shortcomings.  The move is
  primarily conceptual, not technical, having to do w/ IVs.
}

\begin{frame}{Example 2: Did ideological innocence end in 1964?}
  \begin{enumerate}
  \item Converse (1964): political ideology, workings of gov't
    bewildering to ordinary Americans.
  \item Nie, Verba and Petrocik (1979): Only before 1964!
  \item Both analyses based on national election \textit{surveys}.
  \item Sullivan, Piereson \& Marcus's (1978)  split ballot experiment
    in Minneapolis-St. Paul
  \end{enumerate}
  
\end{frame}

\subsection{Conceptual framework}
\begin{frame}{Goals of experimental design}

Experiment design aims to create a situation in which causation is
readily detected, in the following sense. First, if the putative causes
really do have their intended effects, they'll manifest themselves in
such a way that the inference of causation is natural to make. Second,
when such inferences suggest themselves they also withstand scrutiny, in
that logic reveals that either the inference of causation was correct or
a freak occurrence, something objectively unlikely, has occurred.

From this perspective, random assignment brough substantial benefits. But there were others\ldots

\end{frame}

\begin{frame}<1>[label=timelineFr]{Timeline, with entry points for biases}
  % commands I'll want to renew, to get slides right in different versions
  \newcommand{\selpt}{ }%{{\fbox{\strut \hspace{2em} \strut}}}
  \newcommand{\biasrow}{}%{ Biases: &   & & & & \\ }

  \begin{center}
    \begin{tabular}{lp{.15\linewidth}p{.15\linewidth}p{.15\linewidth}p{.15\linewidth}p{.15\linewidth}}
      & \multicolumn{2}{c}{\underline{$T< 0$}} & \underline{$T=0$} & \multicolumn{2}{c|}{\underline{$T>
        0$}} \\
& subjects identified & subjects recruited & treatment assigned &
treatment delivered & outcomes  measured\\ \hline
& \fbox{Healthy}   &            &          & & \\
 & \fbox{Sick}        &  \selpt      &     & & \\
 &                           &            &   \fbox{T}  & &  \\
 & \fbox{Urban}       &            &               & \fbox{T} & \\
 & \fbox{Rural}  &  \selpt      &           &  & \\
 &                            &            &                 & \fbox{C} & \\
 & \fbox{Rich}         &            &    \fbox{C} &  & \\
 & \fbox{Poor}        &  \selpt      &             &  & \\
 &   (etc)                        &            &                &  \\ 
\biasrow% T unlike C at beginning of study, 
                        %sample not rep of pop,
                        % differential uptake, placebo effects,
                        % differential attrition
   \end{tabular}
  \end{center}
\end{frame}

\begin{frame}{Potential outcomes}
  
  \begin{itemize}[<+->]
  \item Distributions of response under treatment and control: $\mu_{t},
    \mu_{c}$; $\mu_{t}(x), \mu_{c}(x)$.
  \item Counterfactual conditional.
  \item $x$ or $\mathbf{x}$?\ldots
  \item Unit-level potential outcomes (Rubin, 1970s): $y_{t}, y_{c}$.
  \end{itemize}

The Rosenbaum reading covers this material in detail, in context of an experiment.
\end{frame}

\begin{frame}<1-3>[label=quasivstrueFr]{Quasi-experiments as
    experiments?}
  \begin{itemize}[<+->]
  \item Controlled experiments vs found ``experiments''
  \item Found randomization as a hypothesis
  \item Theoretical vs empirical scrutiny of hypothesis
  \item Informal empirical scrutiny
  \item Formal hypothesis testing (instructor agenda alert!)
  \end{itemize}
\end{frame}

\section*{Preview}
\againframe<\nottheirhandout>{whatWeWillCoverFr}

\againframe<4\mynoteonly>{quasivstrueFr}

\begin{frame}{Balance after random assignment}
  \begin{center}
    \igrphx[height=.8\textheight]{rb2010tab21}
  \end{center}
\end{frame}

\begin{frame}{Checking/displaying balance after a PS matching}
  \begin{center}
  \igrphx[height=.8\textheight]{tomlovepic-lb-flat}    
  \end{center}
\end{frame}


\againframe<5\mynoteonly>{quasivstrueFr}

\begin{frame}{Validating an RCT via balance checking}
  \begin{center}
    \igrphx[height=.8\textheight]{hb09jasa-fig3}
  \end{center}
\end{frame}

\begin{frame}{Course prerequisites}
  \begin{enumerate}
  \item mathematical proofs, enough to distinguish correct from incorrect 
\item experience using a command-based computer program (as opposed to a menu-driven one)
\item willingness to adopt a try-it-and-see-what-happens approach to software
\item willingness to learn R  (if it's new to you, consider taking the course)
\item willingness to learn markdown (you can do this on your own)
\item willingness to learn to read LaTeX

  \end{enumerate}
\end{frame}

\end{document}
\section{Matching in nonrandomized studies}

\begin{frame}[label=CIFr]{``Causal inference'' again}

`Causal inference'' is about design and analysis of randomized and
nonrandomized studies with goals:
\begin{enumerate}
\item Meaningful comparisons, ie comparisons that reveal causation (if present) simply and w/ minimal overhead;
\item On scrutiny, EITHER indicated causation occurred OR \ldots
\end{enumerate}


(where a short, circumscribed list follows the ``OR'')

\end{frame}

\begin{frame}{Example 4: 3 ways to evaluate effectiveness of charter schools }

\begin{enumerate}
\item (The debate)
\item Lottery
\item ``Exact match''
\item Propensity score matching
\item Fortson et al (2012)'s conclusion. (See also Bifulco 2012.)
\item Relating to our characterization of causal inference\ldots
\end{enumerate}
\end{frame}

\begin{frame}{Example 4: Fortson et al (2012), matching vs lotteries}
\framesubtitle{An experimental benchmark}

\igrphx[height=.8\textheight]{fortsonetal-tabiv5a}

\end{frame}


\begin{frame}{Example 4: Fortson et al (2012), matching vs lotteries}
\framesubtitle{Propensity score matches}

\begin{tabular}{cc}
\igrphx[height=.8\textheight,width=.2\linewidth]{fortsonetal-tabiv5b} & 
\igrphx[height=.8\textheight]{fortsonetal-tabiv5c} \\
\end{tabular}
\end{frame}


\begin{frame}{Example 5: Arceneaux et al 2010}
\framesubtitle{Matching to address response bias in GOTV experiments}

\igrphx[height=.8\textheight]{arceneauxetal2010table2}

\end{frame}
\section{recap}

\againframe{CIFr}

\end{document}
